\documentclass[12pt]{article}   % list options between brackets
\usepackage{bibentry}              % list packages between braces
\usepackage{url}              % list packages between braces
\usepackage{parskip}
\usepackage{newtxtext}
\usepackage{setspace}
\usepackage{apacite}

\begin{document}

\onehalfspacing % Set line spacing to 1.5

\title{%
    FAFRIL\\
    \smallskip
    \Large Rapport av friluftsvistelse
}   % type title between braces
\author{Erik Hedlund}         % type author(s) between braces
\date{\today}    % type date between braces
\maketitle

\section*{Del 1}

% Motivera och förklara med hjälp av litteraturen, varför du valt att genomföra
% friluftsvistelsen på detta sätt i relation till friluftslivets
% innebörd och dess betydelse, hur den kan ses som hållbart friluftsliv och hur
% dess innehåll öppnar för möjligheter till livslångt lärande.
% Exemplifiera tydligt.
%
% (Minst 600 ord max 1050 ord.)

Som en person med måttlig erfarenhet av friluftsliv hade jag ett mål med denna
friluftsvistelse som planeringen kretsade runt; \textbf{att genomföra en aktivitet
jag aldrig tidigare gjort.}

Att konstant söka nya utmaningar skulle kunna påstås vara ett grundsten inom idrott
och friluftsliv då det krävs för att väga upp för ens utvecklande färdigheter och bibehålla
utmaningen\cite{arnegaard2006}. Jag skulle kunna påstå att det var anledning för
ovanstående mål, men jag ska erkänna att detta hade varit en efterhandskonstruktion
och den sanna anledningen för detta val var att undvika leda i en hobby som annars lätt kan bli
repetitiv och monoton om man låter den. Från ett individualistiskt hållbarhets perspektiv
skulle man kunna argumentera för att just variation inom hobbyn är en nödvändighet för
ett långvarigt intresse, då utbränning annars riskerar att ske.

Med målet av att göra något nytt bestämt var jag tvungen att konkretisera
en faktisk exkursion. I samband med att jag läste ``Friluftssport och äventyrsidrott''
fick jag upp ögonen för det mer äventyrliga friluftslivet, det vill säga aktiviteter
som karakteriseras av sin ovisshet\cite{sandell2011friluftssport}.

Den äventyrliga aktivitet jag konstruerade utgjordes av en utmaning till mig själv;

\begin{enumerate}
        \item Jag ska vandra fritt utanför nationalpark/vandringsled tills att jag hittar en bra/idyllisk plats att slå läger
        \item Dagen efter ska jag hitta tillbaka utan hjälp utav karta/redskap.
\end{enumerate}

Anledningen till att ställa en utmaning till mig själv och senare bygga upp en plan
för att genomföra den, snarare än att bara planera än resa var för att försöka
skapa en möjlighet till problembaserat lärande.

Med facit i hand kan jag dock säga att
både stå för problemet och lösningen av det inskränkte nog på metodens effektivitet, då det som
person är svårt att skapa problem/utmaningar till sig själv som kretsar
runt kompetenser man ej besitter, just för att det är lätt att veta vad man vet
och betydligt svårare att veta vad man inte vet. Att jag även gjorde utmaningen individuellt
istället för i grupp (då basgruppen är en integral del av PBL\cite{silen2004} av en anledning)
försämrade nog inlärningsmöjligheten väsentligt. Då jag ej hade några gruppkamrater
att lära mig från blev det största inlärningsmomenten för utmaningen i ``trial and error'' stuk.

Valet att göra utflykten ensam var dock inte omotiverat. En stor anledning till detta
var för att primära syftet med exkursionen var kroppsupplevelsen som definerad av \cite{sandell2011friluftssport},
det vill säga upplevelsen av naturen skulle stå i fokus och jag ville
inte sociala intryck skulle distrahera från detta. Sen fanns det även
än mindre prestationsmoment (även definerad av \cite{sandell2011friluftssport})
i det hela där jag vill försäkra mig om att jag åstadkom var något jag klarat
på egen hand. Dock nu i efterhand kan jag tycka att det inte var någon större prestation
i det jag gjorde, då ingenting med att tälta/vandra ensam skiljde sig markant från
att göra det i grupp.

Valet att inte använda karta gjordes gjordes också först och främst för utmaningens skull,
men även det hade bieffekten av att det var lättare att fullfölja kroppsupplevelsen och
fokusera på skogen och stunden\cite{loynes2020}. På grund av säkerhetsskäl fanns dock en karta
fanns nedpackad i väskan ifall krisen skulle vara framme. Lyckligtvis behövde den aldrig användas.

Efter att utmaningen bestämts återstod att välja plats för aktiviteten. Skogen mellan
min barndoms hemby och närmsta grannort slog mig som perfekt av flera anledningar:

\begin{enumerate}
        \item Den har traditionellt inte använts för friluftsliv, så den är relativt orörd av människan.
        \item Det finns inga stigar (till min vetskap) mellan byn och orten som går över skogen\footnotemark{}\footnotetext{Det finns en gammal skoterled som inte verkar varit underhållen på flera år och nog inte går att följa}
        \item Inga farliga rovdjur bor i skogen.\footnotemark{}\footnotetext{I samband med att jag skriver den här texten undersökte jag det här och har upptäckt att rovdjursattacker på människor är nästintill icke existerande i Sverige\cite{om_rovdjuren}, men det visste jag inte då.}
        \item Den är belägen på en halvö, så håller man sig i en riktning kommer man antingen ut ur skogen eller till vatten. Det vill säga det är lätt att hitta ut om man villar bort sig helt.
        \item Skogen var måttligt stor, med 827 hektars area.
\end{enumerate}

Att skogen låg inom gångavstånd från mitt boende var också till fördel på några punkter:
Dels innebar det att skogsområdet tekniskt sett kunde anses vara tätortsnära (även om min hemby per definition
inte är en tätort), något som är indikativt av att aktiviteten är mer hållbar ur ekologisk och socialt perspektiv.
Tätortsnära naturområden beskrivs som områden man inom rimligt
tid kan resa till under en vardag\cite{fredman2013friluftsliv}, något som kan anses vara en grundpelare för att
regelbundet kunna utöva friluftsliv. Jag hade utan svårigheter kunnat
besöka den här skogen under en vardag, och faktum är att jag ofta promenerar där det när jag är i byn.

Denna exkursion var även hållbar ur ett ekologisk perspektiv. En uppenbar orsak till detta var
att ingen resa med bil eller kollektivtrafik behövde göras, men andra val, så som att inte
införskaffa sig ny utrustning och bara nyttja redan ägd/låna det som saknas för att minska
konsumtion också. Exkursionen planerades även in under en period då jag skulle vara i hembyn
oavsett så att ingen resa mellan Kramfors och Stockholm gjordes enbart för aktivitetens skull.
Skulle man vilja kvantifiera det hela kan man påpeka att utflykten följde samtliga av
relevant punkter för hållbart friluftsliv enligt \cite{checklista}.

Sedan ska jag även erkänna att exkursionen självklart inte var helt utan ekologisk påverkan.
Skulle många fler göra som jag och börja vandra i denna skog skulle dess relativa orördhet
försvinna och därmed omöjliggöra denna typ aktivitet. Man kan därför argumentera för att en premiss
för denna aktivitets hållbarhet är att den inte görs av många, vilket kan anses vara ett egoistisk
perspektiv på friluftsliv.

\section*{Del 2}

% Problematisera och kritiskt reflektera, med hjälp av litteraturen utifrån motiven
% i punkt 1, över hur friluftsvistelsens innehåll och de
% tre begreppen relaterar till och påverkar varandra. Exemplifiera tydligt.
%
% (Minst 400 ord max 500 ord)

Att relatera friluftslivets innebörd, hållbart friluftsliv och livslångt lärande tillsammans var
något som inte tedde sig helt uppenbart i början av den här uppgiften men som har blivit klarare
längst med skrivandet av del 1.

Inledningsvis skulle jag vilja motivera vikten av hållbart friluftsliv har för livslångt lärande.
Det säger ju sig självt att för att kunna lära livet ut av friluftsliv måste möjligheten samt
glöden för att praktisera hobbyn inte försvinna med tiden. Just när det gäller livslångt
lärande tror jag att den individuella hållbarheten i friluftsliv är mest relevant för diskussion.
Även om risken finns att vi som människor förstör naturen till den grad att det ej längre
är möjligt för någon att praktisera friluftsliv, finner jag det mycket mer troligt att man som individ
kommer brännas ut mentalt eller fysiskt innan det sker.

Att göra någon ny form a friluftsliv
i samband med den här uppgiften gjordes just för att undvika uttråkning av hobbyn, som nämnt
ovan, då jag vet av tidigare erfarenheter att det är lätt hänt. Jag är ännu inte rädd för att slita
ut kroppen i samband med hobbyn än så länge då jag är ung och kry, men det är något jag även
kommer behöva ta i beaktning i framtiden.

Livslångt lärande och friluftslivets innebörd relaterar till varandra genom att filosofierna
inom friluftslivet kan hjälpa en konkretisera sina tankar och erfarenheter, för att sedan
lära sig av dem. Hade jag till exempel inte känt till filosofierna om till exempel kroppskulturer
hade jag haft väsentligt svårare att designa en aktivitet som denna som kretsar
kring just en en av de kulturerna. Likväl hade det i efterhand varit svårt att
reflektera över upplevelsen mer avancerat än ``jag var ute i skogen och villade fram
och tillbaka i några timmar, det var kul.''

För att dra paralleller till något mer traditionellt och, beroende på vem man frågar, hårt
filosofiämne tänker jag jämföra med matematiken: En person bekant med de grundläggande räknesätten
skulle egentligen kunna härleda resterande satser och beräkna de mest svåra matematiska
uttryck på egen hand. Dock så kommer personen med god förståelse för matematikens historik och
och som är påläst om dessa kända formler och satser kunna göra detta mycket enklare, samt
även förstå vad den gör mycket djupare.

Hur friluftslivets innebörd och hållbart friluftsliv relaterar är dock något jag finner mindre
uppenbart. Om man anser dessa relationer vara transitiva, det vill säga om \textbf{A} relaterar \textbf{B} och
\textbf{B} relaterar \textbf{C} anser vi att \textbf{A} relaterar \textbf{C}\cite{wiki:transitive}. Det innebär, mindre formellt
uttryckt, att eftersom hållbart friluftsliv relaterar till livslångt lärande och livslångt
lärande till friluftslivets innebörd (eller andra vägen om), relaterar även hållbart
friluftsliv och friluftslivets innebörd till varandra.

Skulle man inte vilja förlita sig på det transitiva argumentet så kan man anse att förståelse
för friluftslivet innebörd ger en den motivering som krävs för att praktisera hållbart friluftsliv,
då man värnar om aktivitetens möjlighet. Djupare än så kan jag inte reflektera över denna relation
utan att spränga ordgränsen.

\pagebreak

\renewcommand{\refname}{Källor}
\bibliography{sources}
\bibliographystyle{apacite}

\end{document}
